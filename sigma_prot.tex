



Many zero-knowledge proof protocols (e.g., Schnorr proofs) fall in the so-called framework of \emph{sigma protocols} for \emph{hard relations}. 
A hard relation is one where there is an efficient algorithm to generate pairs $(x,w)$, but it is computationally infeasible to compute $w$ given only $x$.
A sigma protocol is a type of proof of knowledge protocol between a prover $\PP$ and a verifier $\VV$, where the prover wants to convince the verifier that for some statement $x$ known to $\PP$ and $\VV$, he knows a witness $w$, such that $(x,w) \in \Rela$, for some relation $\Rela$. Informally, we want the sigma protocol to be complete, which means that an honest prover can always convince the verifier that he knows a witness; we want the sigma protocol to be sound, which means that if a prover does not know a witness, he can only convince the verifier with a limited probability, and we want the protocol to be zero-knowledge, which means that the prover does not reveal anything about the witness.
%, except the fact that he knows a witness $w$ such that $(x,w) \in \Rela$. 
More formally, a sigma protocol is defined as follows: 

\begin{definition} \label{def:sigmaprot}
We say $\Sigma = (P_1,P_2,V_2)$ is a sigma protocol with challenge space $\CC$, for a relation $\Rela$ if the following properties are satisfied:

\begin{itemize}
    \item {\bf Three-round public coin.} The prover $\PP$ starts the protocol by generating a first message $\com \gets P_1(x,w)$ (often called a commitment), and sending it to the verifier $\VV$. Then, the verifier chooses a challenge $\chall \gets \CC$ uniformly at random from the challenge space $\CC$, and sends it to $\PP$. Finally, $\PP$ computes a response $\resp \gets P_2(\chall)$ ($P_1$ and $P_2$ share a state) and sends it to $\VV$, who runs the verification algorithm $V_2(x,\com,\chall,\resp)$, which outputs $1$ or $0$, signalling that $\VV$ accepts or rejects the proof respectively. (See \cref{fig:sigmaprotocol})
    
    \item {\bf Completeness/correctness.} The completeness (sometimes called correctness) property says that if both parties follow the protocol, and if the prover uses a valid witness $w$ such that $(x,w) \in \Rela$, then the verifier will accept the proof with probability 1. I.e., for all $(x,w) \in \Rela$, we have \[
\Pr \left[ V_2(x,\com,\chall,\resp) = 1 \middle| \begin{array}{l} \com \gets P_1(x,w) \\ \chall \gets \CC \\ \resp \gets P_2(\chall) \end{array} \right] = 1 \, . 
\]

    \item {\bf $n$-Special Soundness.} A sigma protocol has $n$-special soundness if there exists an efficient extractor algorithm that given $n$ transcripts of the form $(x,\com,\chall_i,\resp_i)$ for $1 \le i \le n$, with $V_2(x,\com,\chall_i,\resp_i) =  1$ for all $i$, and $\chall_i \neq \chall_j$ for all $1 \le i < j \le n$, outputs a witness $w$ such that $(x,w) \in \Rela$. 
    
    \item {\bf Special honest-verifier zero-knowledge.} A sigma protocol is called special honest-verifier zero-knowledge if there exists an efficient simulator $\Sim$ that given $x$ and $\chall \in \CC$ outputs transcripts $(x,\com,\chall,\resp)$ that are indistinguishable from transcripts of honest executions of the protocol. More precisely, for every $(x,w) \in \Rela$, and every $\chall \in \CC$ we require that \[ 
\Sim(x,\chall) \approx \left\{ (x,\com,\chall,\resp) \middle| \begin{array}{l} \com \gets P_1(x,w) \\ \resp \gets P_2(\chall) \end{array} \right\} \,. \] 
If the distributions are identical, we say the sigma protocol has \emph{perfect} special honest-verifier zero-knowledge (sHVZK), if the distributions are statistically close, we say the protocol has \emph{statistical} sHVZK, and if the distributions are only computationally indistinguishable, then we say the sigma protocol is \emph{computationally} sHVZK.
\end{itemize}

\end{definition} 

Section 4.3 of Goldreich~\cite{Gol01} gives a stronger formulation of computational zero knowledge that considers malicious verifiers. 
%\CP{is this useful to mention?}

\begin{figure}
    \centering
    \begin{adjustbox}{minipage=.95\linewidth,fbox,center}

    \begin{tabularx}{\textwidth}{bsb}
    \multicolumn{1}{c}{{\bf Prover}($x,w$)} &  & \multicolumn{1}{c}{{\bf Verifier}($x$)} \\
    \\
    \quad $\com \gets P_1(x,w)$ & & \\
     &  \multicolumn{1}{c}{ $\xrightarrow{\quad \com \quad} $ }  & \\
     & & \quad $\chall \gets \CC$ \\
     & \multicolumn{1}{c}{ $\xleftarrow{\quad \chall \quad } $ } & \\ 
    \quad $\resp \leftarrow P_2(\chall)$ & & \\
    & \multicolumn{1}{c}{ $\xrightarrow{\quad \resp \quad }$} & \\
    & & \quad {\bf Return} $V_2(x,\com,\chall,\resp)$  \\
    \end{tabularx}
    \end{adjustbox}
    \caption{The structure of a general sigma protocol.}
    \label{fig:sigmaprotocol}
\end{figure}

\begin{remark}
The existence of the simulator implies that, for any sigma protocol that is special honest-verifier zero-knowledge, there is always a cheating prover who does not know a witness but who guesses the challenge and can respond correctly if their guess is correct.
Hence one can always cheat with probability at least $1/|\CC|$.
We call the successful cheating probability the \emph{soundness error}.
A protocol with soundness error $\epsilon$ should be computed $k$ times, so that the overall soundness error $\epsilon^k$ is negligible (usually $< 2^{-128}$). 
%\CP{I think for the GPS paper we had a remark that this soundness amplification procedure works when a basic protocol is 2-special sound, but may not be proven secure in general. Steven, does that sound familiar to you?} \SG{No, the text here is correct. The soundness error $\epsilon$ is not necessarily equal to $1/|\CC|$. In our paper it was $(|\CC|-1)/|\CC|$. That's why the next sentence says:}
The ideal case is when the soundness error is equal to $1/|\CC|$ and $|\CC|$ is large.

 Suppose the sigma protocol is $n$-special sound where $n > 2$. Suppose further that, for a statement $x$ and a commitment $\com$, a cheating prover can respond to $n$ distinct challenges. Then the prover can generate $n$ transcripts, and by the $n$-special soundness property can compute a witness $w$ such that $(x,w) \in \Rela$. It follows that the cheating prover must only be able to answer up to $n-1$ distinct challenges. Hence if a protocol is $n$-special sound then the soundness error should be at most $(n-1)/|\CC|$.
\end{remark}

\begin{remark}
The special honest-verifier zero-knowledge property implies that an honest execution of the protocol does not reveal any additional information about $w$, because everything that could be learned by inspecting a transcript $(x,\com,\chall,\resp)$ could also be learned directly from $x$ alone, by first running the simulator $\Sim(x,\chall)$ and then inspecting the simulated transcript. 
\end{remark}

\begin{remark}
Perfect ZK and statistical ZK provide an information-theoretic guarantee that a verifier learns nothing about the witness from the proof. Hence they are preferable to computational ZK.
Computational ZK usually requires introducing new computational assumptions (typically decisional assumptions) beyond what is needed for the hardness of the relation, and this leads to increased complexity and possible weaknesses.
The sigma protocols in the case of CSIDH have perfect or statistical ZK, but the sigma protocols for SIDH that we describe in Section~\ref{sec:SIDH-setting} and Section~\ref{sec:GPSandSQIsign} rely on computational assumptions. \CP{I partly disagree with what is written here: Section 6 has nothing to do with SIDH;  ZK proof underlying GPS has statistical ZK; statistical ZK for some CSIDH-based protocols only conjectural} \SG{Agree this remark needs to be tweaked. But the main point is correct, that perfect ZK is better than statistical ZK is better than comput ZK.}
\end{remark}




\iffalse

Definitions n-special soundness and computational ZK.


Reference: Damg{\aa}rd's 2010 notes

Let $\lambda$ be a security parameter.
Let $R = R_\lambda$ be a binary relation on $(\{ 0,1 \}^\lambda )^2$. If $(x,w) \in R$ then $w$ is called the \textbf{witness} for $x$. ($w$ may not be unique.)
A relation $R_\lambda$ is \textbf{hard} if there is a probabilistic polynomial-time (in $1^\lambda$) algorithm that outputs a pair $(x, w) \in R_\lambda$, and if there is no efficient algorithm that given $x$ computes $w$.


\textbf{Sigma protocol}: Prover $P$ and verifier $V$ both know $x$. Prover sends commitment $\com$. Verifier responds with challenge $\chall$. Prover sends response $\resp$. Verifier runs algorithm
$Ver( x, \com, \chall, \resp )$ which returns 0 (reject) or 1 (accept).


\textbf{Soundness} (also called 2-special soundness):
There exists a polynomial-time algorithm (called an extractor) that takes $( x, \com, \chall, \resp , \chall', \resp')$ such that $\chall' \ne \chall$
$Ver( x, \com, \chall, \resp ) = 1 $ and $Ver( x, \com, \chall', \resp' ) = 1$ and returns a witness $w$.

\textbf{Honest-verifier zero-knowledge}:
There exists a polynomial-time simulator $M(x,\chall)$ that outputs $(\com, \resp)$ such that $Ver( x, \com, \chall, \resp ) = 1 $.

This is \textbf{statistical} HVZK if the transcripts $(\com, \chall, \resp )$ are distributed identically to transcripts in the honest protocol.

\textbf{Computational} HVZK is when transcripts are computationally hard to distinguish from the honest protocol.


Precise definition of Computational ZK:
(Section 4.3 of Goldreich~\cite{Gol01})
For an element $x \in L$ we define the transcript of the interaction between an honest prover $P$ and a potentially dishonest verifier $V^*$ by $view^P_{V^*}(x)$.
An interactive proof is computational zero knowledge if for every ppt algorithm (malicious verifier) $V^*$ there is a ppt simulator $M^*$ such that 
for all negligible functions $negl(n)$ and all distinguishers $A$ and all sufficiently large $n$ and all instances $x \in L \cap \{ 0,1 \}^n$ of the language
we have
\[
   \Pr( A( view^P_{V^*}(x) )=1 ) - \Pr( A( M^*(x) )=1 )]  = negl(n).
\]



\textbf{Witness indistinguishable} (WI): the distribution of transcripts $(\com, \chall, \resp )$ is independent of the choice of witness.

Q: Is there such a thing as computational witness indist???

Q: Is there any point to discussing this?

\fi
